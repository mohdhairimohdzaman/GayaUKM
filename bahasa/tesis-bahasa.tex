%% Contoh tesis GayaUKM dalam Bahasa Melayu
\documentclass[bahasam,nohyphen]{GayaUKM}

\usepackage{graphicx}

\title{<Tajuk Tesis Anda, Pecahkan Tajuk Panjang Secara Manual\protect\\Jika Perlu>}
\author{<Nama Anda>}
\authorid{<P00000>}
\faculty{<Fakulti Anda>}
\submissiondate{2 Oktober 2013}
\submissionyear{2013}
\degreetype{Doktor Falsafah}
\campus{Bangi}

%% If you find the boxes around hyperlinks distracting
\hypersetup{colorlinks,allcolors=black}

\begin{document}

%% Cover page
\makecoverpage

%% Re-specify your title with different manual line breaks for the
%% title page, if necessary
\title{<Tajuk Tesis Anda, Pecahkan Tajuk Panjang\protect\\Secara Manual Jika Perlu>}
\maketitlepage

\frontmatter
\declaration

% penghargaan dari penghargaan.tex
\input{penghargaan}

% abstrak dlm Bahasa Melayu dari abstrak-ms.tex
\input{abstrak-ms}

% abstrak dlm Bahasa Inggeris dari abstract-en.tex
\input{abstract-en}


\tableofcontents\clearpage
\listoffigures\clearpage
\listoftables\clearpage

% Senari simbol dll boleh disediakan seperti
% dalam senaraisimbol.tex
\chapter{Senarai Simbol}

\begin{center}
\doublespacing
\begin{tabular}{l@{\hspace{3em}}p{.6\textwidth}}
$b, c$ & pemalar\\
$C_f$ & pekali geseran kulit setempat\\
\end{tabular}
\end{center}



\mainmatter
% Setiap satu bab dari fail berasingan
\input{bab-pengenalan}
\input{bab-latarbelakang}
\input{bab-fibonacci}
\input{bab-goldenratio}

% rujukan tersenarai dlm refs.bib
\bibliography{refs}

\appendix
% Setiap satu bab apendiks dari fail berasingan
\input{ap-huraian}
\input{ap-perisian}
\end{document}
