%% Example GayaUKM thesis in English
\documentclass[english,nohyphen]{GayaUKM}

\usepackage{graphicx}
\usepackage{longtable} %% If your list of symbols, abbreviations is longer than a page and you want to use a longtable for it

%% Your metadata. Note that an English thesis needs a Malay title page as well,
%% so you'll need to specify the Malay title, faculty and degreetype.
\title{<Your Thesis Title, Manually Break\protect\\the Lines into a Nice Reverse\protect\\Pyramid if Necessary>}
\titlems{<Tajuk Tesis Anda, Pecahkan Tajuk Panjang\protect\\Secara Manual Jika Perlu>}
\author{<Your Name>}  %% Assuming your name is spelt the same way in English and Malay
\authorid{<P00000>}
\faculty{<Your Faculty>}
\facultyms{<Faculti Anda>}
\submissiondate{2 October 2013}
\submissionyear{2013}
\degreetype{Doctor of Philosophy}
\degreetypems{Doktor Falsafah}
\campus{Bangi} %% Assuming Malaysian cities are spelt the same way in English and Malay

%% If you find the boxes around hyperlinks distracting
\hypersetup{colorlinks,allcolors=black}

\begin{document}

%% Generate the cover page
\makecoverpage

%% Generate the English and Malay title pages.
%% Re-specify your title with different manual line breaks for the English
%% title page, if necessary
\title{<Your Thesis Title, Manually Break the Lines into a Nice Reverse Pyramid if Necessary>}
\maketitlepage

%%%% If you're happy to just use the same line-breaking scheme for the English
%%%% title on both the cover and the titlepage, then you can just call
%%%% \maketitle which combines both \makecoverpage and \maketitlepage.

\frontmatter
\declaration

% Acknolwedgements from ack.tex
\input{ack}

% English abstract from abstract-en.tex
\input{abstract-en}

% Malay Abstract from abstrak-ms.tex
\input{abstrak-ms}


\tableofcontents
\listoffigures
\listoftables

% List of Symbols may be prepared as in symbols.tex
\chapter{List of Symbols}
\begin{center}
\doublespacing
\begin{tabular}{l@{\hspace{3em}}p{.6\textwidth}}
$b, c$ & constants\\
$C_f$ & local friction coefficient\\
\end{tabular}
\end{center}



\mainmatter
% Each chapter from a separate file
\input{chap-intro}
\input{chap-fibonacci}
\input{chap-goldenratio}



% references are listed in refs.bib
\bibliography{refs}

\appendix
% Each appendix chapter from a separate file
\input{app-details}
\input{app-code}
\end{document}
